Elixir jest nowym językiem programowania opartym o dorobek języka
Erlang, stworzonego przez firmę Ericsson. Jest dynamicznie typowany,
utrzymany w funkcyjnym paradygmacie funkcyjnym, zaprojektowanym do
tworzenie skalowalnych, łatwych w utrzymaniu aplikacji.
\cite{thomas2014elixir} Wykorzystanie Erlanga jako podstawy nie
sprowadza się jedynie do założeń. Erlang to nie tylko funkcyjny język
programowania, ale również zestaw wzorców projektowych, zwanych OTP
(dawniej Open Telecom Platform \cite{logan2010erlang}), oraz wirtualnej
maszyny Erlanga, BEAM. Ta ostatnia została stworzona od podstaw w celu
wspierania złożonych, współbieżnych i rozproszonych systemów o wysokiej
odporności na błędy. Kod Elixira, jest wykonywany w ramach lekkich
procesów (aktorów), trzymających własny stan i komunikujących się między
sobą poprzez przesyłanie wiadomości (ang. \emph{message passing}).
Wirtualna maszyna sama rozdziela procesy pomiędzy rdzenie procesora
zapewniając równoległość przetwarzania oraz dbając o żywotność procesów.
Elixir odziedziczył również całe zaplecze narzędzi stworzonych na
potrzeby Erlanga oraz kod biblioteczny, który może być wykorzystywany
bez strat wydajności. Nie jest to jedynie próba odświeżenia 30-o
letniego języka jakim jest Erlang, a wzbogacenie go przydatne
funkcjonalności, jak metaprogramowanie czy polimorfizm, oraz przyjazną
składnię. Celem twórcy Elixira, José Valima, było stworzenie
rozszerzalnego, przyjaznego programistom języka. \cite{valim2013design}
