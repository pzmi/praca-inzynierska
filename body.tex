\section{Wstęp}\label{wstux119p}

Cały wstęp stanowi opis tematu pracy, streszczenie treści.

\subsection{Opis dziedziny}\label{opis-dziedziny}

\subsection{Cele pracy}\label{cele-pracy}

\subsection{Zakres pracy}\label{zakres-pracy}

\section{Opis technologii}\label{opis-technologii}

Języki programowania to nie tylko składania, ale cały ekosystem
technologii tworzony wokół nich.

\subsection{Java}\label{java}

Java jako standard aplikacji biznesowych. Przedstawienie powszechności
Javy w biznesie, współczesny Cobol.

\subsection{JavaScript}\label{javascript}

Opis popularności JavaScript, języka stworzonego na cele interfejsu
użytkownika, po stronie serwerowej. Wykorzystanie JS dla backendu dzięki
Node.js. Założenia i podstawy Node.js.

\subsection{Elixir}\label{elixir}

Opis Elixira - młody, zyskujący na popularności popularności język,
nowatorskie podejście oparte na silnych podstawach Erlanga.

\section{Architektura systemu
informatycznego}\label{architektura-systemu-informatycznego}

Typowe wzorce architektoniczne dla poszczególnych technologii.

\subsection{Java}\label{java-1}

Architektura silnie związana ze standardami JavaEE.

\subsubsection{Architektura
wielowarstwowa}\label{architektura-wielowarstwowa}

Standardowy wzorzec dla aplikacji JEE. Opis podziału na część webową,
logikę biznesową i bazę danych.

\subsection{JavaScript}\label{javascript-1}

Brak standardowych rozwiązań. Przedstawienie popularnych podejść z
wielu, wynikających z szerokiej społeczności.

\subsubsection{Asynchroniczność}\label{asynchronicznoux15bux107}

libuv jako podstawa działania Node.js. Wady i zalety asynchroniczności
dla modelu przyjętego w Node.js

\subsection{Elixir}\label{elixir-1}

Dziedzictwo 30 lat Erlanga i wypracowanych wzorców. Standardy i wzorce
OTP.

\subsubsection{Model aktorowy}\label{model-aktorowy}

Opis modelu aktorowego zakorzenionego w wirtualnej maszynie Erlanga.
Popularność modelu aktorowego w ostatnich latach na przykładzie
biblioteki Akka.

\subsection{Współczesne wzorce
architektoniczne}\label{wspuxf3ux142czesne-wzorce-architektoniczne}

Aplikacje ``w chmurze'', rozproszone aplikacje, architektura
mikroserwisowa, REST.

\section{Wydajność}\label{wydajnoux15bux107}

Implementacja kilku aplikacji w każdej z przedstawionych technologii.
Test obciążeniowy każdej z implementacji w celu porównania osiągów.

\subsection{Proste zapytanie}\label{proste-zapytanie}

Proste zapytanie zwracające łańcuch tekstowy.

\subsection{Czasochłonne obliczenia}\label{czasochux142onne-obliczenia}

Aplikacja wykonująca bardziej wymagające operacje z większym romiarem
danych, np. obliczanie silni dla dużych liczb, transponowanie macierzy.

\subsection{Ograniczenia
wejścia/wyjścia}\label{ograniczenia-wejux15bciawyjux15bcia}

Wsparcie dla operacji we/wy, np. zapis i odczyt z bazy danych.

\subsection{Wnioski}\label{wnioski}

Porównanie otrzymanych wyników.

\section{Skalowalność}\label{skalowalnoux15bux107}

\subsection{Java}\label{java-2}

Klasteryzacja serwerów aplikacyjnych.

\subsection{JavaScript}\label{javascript-2}

Biblioteki do rozproszonych systemów Node.js.

\subsection{Elixir}\label{elixir-2}

Distributed Erland i Distributed Elixir. Protokoły komunikacji między
węzłami w samym języku.

\subsection{Wnioski}\label{wnioski-1}

\section{Produktywność}\label{produktywnoux15bux107}

Ilość linii kodu i linii konfiguracji. Środowisko programisty,
dostępność narzędzi i bibliotek.

\subsection{Java}\label{java-3}

\subsection{JavaScript}\label{javascript-3}

\subsection{Elixir}\label{elixir-3}

\subsection{Statystyki}\label{statystyki}

Analiza społeczności jako istotnego czynnika rozwoju technologii na
podstawie statystyk z GitHub, Maven Central, npmjs, hex.pm.

\subsection{Wnioski}\label{wnioski-2}

\section{Uwagi końcowe}\label{uwagi-koux144cowe}

\section{Literatura}\label{literatura}
