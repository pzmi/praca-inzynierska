\section{Wstęp}

Cały wstęp stanowi opis tematu pracy, streszczenie treści.

\subsection{Opis dziedziny}

\subsection{Cele pracy}

\subsection{Zakres pracy}

\section{Opis technologii}

\subsection{Java}

Java jako standard aplikacji enterprise. Przedstawienie powszechności
Javy w biznesie, współczesny Cobol.

\subsection{JavaScript}

Wykorzystanie JavaScript po stronie serwerowej dzięki Node.js. Założenia
i podstawy Node.js.

\subsection{Elixir}

Opis Elixir - młody język, nowatorskie podejście oparte na silnych
podstawach Erlanga.

\section{Architektura systemu
informatycznego}

Typowe wzorce architektoniczne dla poszczególnych technologii.

\subsection{Java}

Architektura silnie związana ze standardami JavaEE.

\subsubsection{Architektura
wielowastrwowa}

Standardowy wzorzec dla aplikacji JEE. Opis podziału na część webową,
logikę biznesową i bazę danych.

\subsection{JavaScript}

Brak standardowych rozwiązań. Przedstawienie popularnych podejść z
wielu, wynikających z szerokiej społeczności.

\subsubsection{Asynchroniczność}

libuv jako podstawa działania Node.js. Wady i zalety asynchroniczności
dla modelu przyjętego w Node.js

\subsection{Elixir}

Dziedzictwo 30 lat Erlanga i wypracowanch wzorców. Standardy i wzorce
OTP.

\subsubsection{Model aktorowy}

Opis modelu aktorowego zakorzenionego w wirtualnej maszynie Erlanga.
Popularność modelu aktorowego w ostatnich latach na przykładzie
biblioteki Akka.

\subsection{Współczesne wzorce
architektoniczne}

Rozproszone aplikacje, REST, przede wszystkim mikroserwisy.

\section{Wydajność}

Implementacja kilku aplikacji w każdej z przedstaiownych technologii.
Test obciążeniowy każdej z implementacji w celu porównania osiągów.

\subsection{Proste zapytanie}

Proste zapytanie zwracające łańcuch tekstowy.

\subsection{Czasochłonne obliczenia}

Aplikacja wykonująca bardziej wymagające operacje z większym romiarem
danych, np. obliczanie silni dla dużych liczb, transponowanie macierzy.

\subsection{Ograniczenia
wejścia/wyjścia}

Wsparcie dla operacji we/wy, np. zapis i odczyt z bazy danych.

\subsection{Wnioski}

Porówanianie otrzymanych wyników.

\section{Skalowalność}

\subsection{Java}

Klasteryzacja serwerów aplikacyjnych.

\subsection{JavaScript}

Biblioteki do rozproszonych systemów Node.js.

\subsection{Elixir}

Distributed Erland i Distributed Elixir. Protokoły komunikacji między
węzłami w samym języku.

\subsection{Wnioski}

\section{Produktywność}

Ilość linii kodu i linii konfiguracji. Środowisko programisty,
dostępność narzędzi i bibliotek. Analiza społeczności jako istotnego
czynnika rozwoju technologii na podstawie statystyk z GitHub, Maven
Central, npmjs, hex.pm.

\subsection{Java}

\subsection{JavaScript}

\subsection{Elixir}

\subsection{Statystyki}

\subsection{Wnioski}

\section{Uwagi końcowe}

\section{Literatura}
